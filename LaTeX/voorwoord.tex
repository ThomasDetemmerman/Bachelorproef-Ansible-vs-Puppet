%%=============================================================================
%% Voorwoord
%%=============================================================================

\chapter*{Voorwoord}
\label{ch:voorwoord}

%% TODO:
%% Het voorwoord is het enige deel van de bachelorproef waar je vanuit je
%% eigen standpunt (``ik-vorm'') mag schrijven. Je kan hier bv. motiveren
%% waarom jij het onderwerp wil bespreken.
%% Vergeet ook niet te bedanken wie je geholpen/gesteund/... heeft

Drie jaar geleden begon ik aan de hogeschool in Gent met een duidelijke voorkeur voor informatica. Nu, drie jaar later, is diezelfde passie alleen maar groter geworden. Gedurende mijn studententijd vond ik voldoening in de meeste asptecten die informatica bood waaronder zaken zoals programmeren, artifici\"ele intelligentie en  netwerk- en systeembeheer. Ondanks het feit dat ik al deze zaken interessant vond, was er \'e\'en onderdeel dat geleidelijk mijn voorkeur zou krijgen en dit werd het beheren van servers.
 \newline
Hier werd al snel duidelijk dat goede configuration management tools een absolute meerwaarde konden bieden. Zo herinner ik me nog mijn eerste project waarbij de opdracht was een webserver op te zetten. Ik maakte toen gebruik van Puppet terwijl ik amper de kracht en het potentieel van dit soort technologie\"en begreep. Inmiddels heb ik de kans gehad om hieromtrent uitgebreide ervaring op te doen dankzij mijn docent dhr. Van Vreckem en mijn stage op de VRT. Dit heeft ertoe geleid  ook mijn bachelorproef rond dit fascinerend onderwerp te voeren.

Een goede bachelorproef wordt niet alleen geschreven. Hierbij werd ik ondersteund door heel wat mensen die ik zeer dankbaar ben. Bij deze wil ik dan ook de volgende mensen persoonlijk bedanken voor hun bijdrage.

\begin{center}
	
	
		\large{\textbf{dhr. De Wispelaere Tom}}\newline
	\textit{Bedankt voor het voorzien van uitgebreide feedback, het bedenken van oplossingen en de aanbreng van nieuwe idee\"en.}\newline
	
			\large{\textbf{dhr. De Weirdt Harm}}\newline
	\textit{Bedankt omdat ik bij u terecht kon voor vragen en voor uw mening over de stand van zaken gedurende de bachelorproef.}\newline
	
		\large{\textbf{dhr. Dierick Gerben}}\newline
	\textit{Bedankt voor het interessante gesprek betreffende de veiligheid en risico's omtrent deze techonologie\"en.}\newline
	
		\large{\textbf{dhr. Adams Pieter}}\newline
	\textit{Bedankt omdat ik bij u terecht kon voor technische vragen en de introductie van Ansible Tower.}\newline
	
	\large{\textbf{mevr. Lambrecht Carine}}\newline
\textit{Bedankt voor het verzorgen van het lingu\"istisch aspect van de bachelorproef en bieden van morele steun.}\newline

	

	
	
	
	
		
		
\end{center}