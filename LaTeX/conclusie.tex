%%=============================================================================
%% Conclusie
%%=============================================================================

\chapter{Conclusie}
\label{ch:conclusie}

 De syntax van Puppet  wordt ten opzichte van die van Ansible door velen aanzien als complex. Dit is bovendien niet enkel het geval voor het schrijven van modules, ook het opstellen van een infrastructuur vergt enige oefening. Elke server die met Puppet dient te communiceren moet een Puppetagent draaiende hebben. Deze moet bovendien zodanig geconfigureerd worden dat hij de Puppetmaster kan bereiken. Vervolgens moeten certificaten uitgewisseld worden. De master verstuurt namelijk alle bestanden in \'e\'en keer en vervolgens neemt de agent het van de master over. Een voordeel van deze werkwijze is dat de Puppetmaster minder zwaar belast wordt doordat de Puppetagents zelf een deel van de verantwoordelijkheid dragen. Hierdoor kan Puppet beter overweg bij een groeiende infrastructuur. Puppet is duidelijk ook sneller. Het verschil is vooral opmerkelijk wanneer er niets tot weinig aangepast dient te worden. 

Ansible daarentegen wordt geprezen voor de eenvoudige syntax en infrastructuur. Ook het feit dat Ansible geen agent gebruikt is \'e\'en van zijn paradepaardjes. Hierdoor valt de volledige verantwoordelijkheid onder de bevoegdheid van de master. De resources van de clients worden hierdoor bespaard maar dit kan wel nadelige gevolgen hebben voor de performantie van de master bij grotere infrastructuren. Gezien het feit dat de master (bij Ansible) zich voortdurend moet ontfermen over de clients, worden meerdere kleinere bestanden verstuurd naar de clients in tegenstelling tot Puppet die alles in \'e\'en keer doorstuurt. Ondanks een voortdurende belasting van het netwerk, heeft Ansible op het einde van de rit het netwerk minder belast. Bovendien is er door deze manier van werken een live feedback mogelijk die te volgen is vanop de master. Bij Puppet moet hiervoor ingelogd worden op de desbetreffende client.
%%%



%Puppet heeft niet voor niets meer dan 36.000 bedrijven die hun infrastructuur met behulp van Puppet uitbouwen. Zo blijkt ook uit dit onderzoek waarbij Puppet performanter is dan Anible maar dat Puppet ook beter kan opereren in grote netwerken. Echter tot het bekomen van dergelijke integraties is er nood aan degelijk opgeleid personeel. Het opstellen van deze servers en bijbehorende configuraties vergt een heel karwei.  Ansible is eenvoudig op te zetten en bijhorende clients dienen niet geconfigureerd te worden wat de zaken vereenvoudigd. Bovendien wordt YAML, waarin Ansible code geschreven wordt, door verschillende mensen ervaren als een eenvoudig en gemakkelijk aan te leren syntax. Een technologie kan performant zijn, maar dit houdt niet in dat deze handig is in gebruik.

Aan bedrijven die beschikken over de nodige kennis omtrent Puppet, wordt geadviseerd om bij deze infrastructuur te blijven. Een overschakeling kost geld en werk en bovendien is er geen winst op gebied van performantie en schaalbaarheid.

Bedrijven die geen ervaring hebben met dit soort technologie\"en wordt aangeraden om voor Ansible te kiezen. Ansible is eenvoudig op te stellen en aan te leren. De essentie van dit soort tools is nog steeds het configureren van servers. De eenvoud in syntax moet dan ook gezien worden als belangrijkste factor. Andere aspecten zoals configuratietijden zijn interressant maar blijven bijzaak.

Ook voor bedrijven die zichzelf herkenen in de situatie van de VRT, zoals beschreven is in sectie \ref{sec:methodologie-redenen-omschakeling}, wordt deze stap zeker aangeraden. Dit soort overschakelingen kan geleidelijk verlopen aangezien beide tools kunnen bestaan in dezelfde omgeving (zie \ref{sec:methodologie-verloop-transit}). Bovendien is de overstap relatief eenvoudig zonder veel complexiteit. Het rendement van de overschakeling weegt dus zeker op ten opzichte van het werk dat vereist is voor een overschakelingsproject.

\begin{center}
	\begin{tabular}{ l | c  c  }
	
		 							& Puppet 		   & Ansible 				\\ \hline
Infrastructuur opstellen& & \checkmark \\
Leercurve &						&  \checkmark			\\ 
Configuratietijd   & \checkmark		&\\ 
Schaalbaarheid   & \checkmark		&\\ 
 \hline \hline
		Belasting netwerk &             		 &	\checkmark			 \\ 
	Onafhankelijk van een goede verbinding& \checkmark & \\
		 Geheugengebruik &						&  \checkmark			\\ 
			
	\end{tabular}
\end{center}

