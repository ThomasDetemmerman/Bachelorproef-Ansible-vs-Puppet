


%-------------------------------- acroniemen
\newacronym{CMT}{CMT}{configuration management tool}
\newacronym{dsl}{DSL}{domain specific language}
\newacronym{LDAP}{LDAP}{Lightweight Directory Access Protocol}
\newacronym{JVM}{JVM}{Java Virtual Machine}
\newacronym{VRT}{VRT}{Vlaamse Radio- en Televisieomroeporganisatie}


%--------------------------------- woordenlijst
\newglossaryentry{Git}{
	name = Git,
	text = Git,
	description = Git wordt in het engels beschreven als ‘distributed version control system'. Op deze wijze kunnen meerdere mensen aan hetzelfde project werken. Ieder teamlid kan zijn code toevoegen aan een gemeenschapelijk project
}

\newglossaryentry{Git-branch}{
	name = {Git branch},
	text = {Git branch},
	description = Er kunnen meerdere versies van \'e\'enzelfde project bestaan. Elke versie bevindt zich dan in een andere branch
}
\newglossaryentry{werkpakket}{
	name = {Werkpakket},
	text = {werkpakket},
	description = {Een term binnen dit onderzoek. Ansible vertaalt een configuratie geschreven in YAML naar meerdere Pythonscriptjes. Elk van deze scriptjes wordt pas verstuurt wanneer zijn voorhanger voltooid is. Zo \'e\'en scriptje wordt hier een werkpakket genoemd}
}
%\newglossaryentry{werkpakket}{
%	name = {Werkpakket},
%	text = {werkpakket},
%	description = 
	
%}


\newglossaryentry{submodule}{
	name = {Submodule},
	text = {submodule},
	description = In Git kan een project onderdeel uitmaken van een groter project. Dit wordt geregeld door in het grote project een link aan te maken naar het kleinere project. Dit kleinere project wordt in Git een submodule genoemd
}


\newglossaryentry{deploy}{
	name = Deploy,
	text = deploy,
	description = Een configuratie uitvoeren. In dit rapport is een vaak gebruikt synoniem configureren. Bijvoorbeeld: Puppet voert een deploy uit op server A. Dit betekend dat Puppet server A configureerd
}

\newglossaryentry{deploy}{
	name = Deploy,
	text = deploy,
	description = Een configuratie uitvoeren. In dit rapport is een vaakgebruikt synoniem configureren. Bijvoorbeeld: Puppet voert een deploy uit op server A. Dit betekent dat Puppet server A configureert
}

\newglossaryentry{puppetplugin}{
	name = Plugin,
	text = plugin,
	description = Extra functionaliteiten worden toegevoegd door middel van een plugin. In het geval van Puppet zijn dit functionaliteiten die geschreven worden in Ruby
}

\newglossaryentry{fact}{
	name = Fact,
	text = fact,
	description = {Ansible en Puppet maken gebruik van facts. Dit zijn gegevens die de master nodig heeft van zijn clients. Dit zijn zaken zoals hostname, besturingssysteem, IP adres etc}
}




\newglossaryentry{catalog}{
	name = Catalogus,
	text = catalogus,
	description = Eng: Catalog. Een catalogus is een term uit de Puppetwereld en is in feite een gecompileerde module. Deze bevat de gewenste configuratie voor een specifieke computer. \autocite{puppetdoc}
}

\newglossaryentry{packagemanager}{
	name = {Package manager},
	text = package manager,
	description = Een mechanisme dat het mogelijk maakt om software te installeren op UNIX gebaseerde systemen \textit{(voorbeelden: yum, apt, dpkg,..)} 
}

\newglossaryentry{configuratietijd}{
	name = {Configuratietijd},
	text = configuratietijd,
	description = {De tijd die de configuration management tool nodig heeft tot het bekomen van een volledig geconfigureerde server}
}

\newglossaryentry{connectietijd}{
	name = Connectietijd,
	text = connectietijd,
	description = {De tijd die het kost alvorens er effectief overgegaan kan worden tot configureren. Hieronder zitten zaken zoals het opstellen van een verbinding, het verzamelen van de nodige gegevens en het versturen van een gepersonaliseerde configuratie}
}

\newglossaryentry{partialconfig}{
	name = {Gedeeltelijke configuratie},
	text = gedeeltelijke configuratie,
	description = {Er is reeds een configuratie aanwezig op de server maar deze is niet meer up-to-date. Bijgevolg moet er een deel opnieuw geconfigureerd worden}
}

\newglossaryentry{fork}{
	name = Fork,
	text = fork,
	description = Het aanmaken van een child process door zichzelf te dupliceren \autocite{forkmeaning}
}
\newglossaryentry{programmeerparadigma}{
	name = Programmeerparadigma,
	description = {Synoniemen zijn ook programmeerstijl of programmeermodel. Voorbeelden zijn object-georienteerd, procedureel, imperatief... \autocite{journalofinformation} }
}

\newglossaryentry{push}{
	name = Push,
	text = push,
	description = {Een manier van communiceren waarbij de actie gestart wordt vanuit een centraal punt, de zender \autocite{pushpullmeaning}}
}

\newglossaryentry{pull}{
	name = Pull,
	text = pull,
	description = {Een manier van communiceren waarbij de actie gestart wordt vanuit de clients, de ontvangers \autocite{pushpullmeaning}}
}

\newglossaryentry{adhoccommando}{
	name = Ad-hoc commando,
	text = ad-hoc commando,
	description = {Een ad-hoc commando is een taak die snel uitgevoerd moet worden maar die niet opgeslagen wordt voor later gebruik \autocite{adhoc}. Het is een eenmalig commando dat geen deel uitmaakt van een groter geheel zoals een playbook}
}

%	description = 

%	description = {Een ad-hoc commando is een taak die snel uitgevoerd moet worden maar dat niet opgeslagen wordt voor later \autocite{adhoc}. Het is een eenmalig commando dat geen deel uitmaakt van een groter geheel zoals een playbook.}

\printglossary[type=\acronymtype,title={Lijst van acroniemen}]
\addcontentsline{toc}{chapter}{\textcolor{maincolor}{Lijst van acroniemen}}
\printglossary
\addcontentsline{toc}{chapter}{\textcolor{maincolor}{Verklarende woordenlijst}}




