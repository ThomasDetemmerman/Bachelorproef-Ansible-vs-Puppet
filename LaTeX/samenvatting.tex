%%=============================================================================
%% Samenvatting
%%=============================================================================

%% TODO: De "abstract" of samenvatting is een kernachtige (~ 1 blz. voor een
%% thesis) synthese van het document.
%%
%% Deze aspecten moeten zeker aan bod komen:
%% - Context: waarom is dit werk belangrijk?
%% - Nood: waarom moest dit onderzocht worden?
%% - Taak: wat heb je precies gedaan?
%% - Object: wat staat in dit document geschreven?
%% - Resultaat: wat was het resultaat?
%% - Conclusie: wat is/zijn de belangrijkste conclusie(s)?
%% - Perspectief: blijven er nog vragen open die in de toekomst nog kunnen
%%    onderzocht worden? Wat is een mogelijk vervolg voor jouw onderzoek?
%%
%% LET OP! Een samenvatting is GEEN voorwoord!

%%---------- Nederlandse samenvatting -----------------------------------------
%%
%% TODO: Als je je bachelorproef in het Engels schrijft, moet je eerst een
%% Nederlandse samenvatting invoegen. Haal daarvoor onderstaande code uit
%% commentaar.
%% Wie zijn bachelorproef in het Nederlands schrijft, kan dit negeren en heel
%% deze sectie verwijderen.



%%---------- Samenvatting -----------------------------------------------------
%%
%% De samenvatting in de hoofdtaal van het document

\chapter*{Samenvatting}
% Nood: waarom moest dit onderzocht worden?
Puppet is een vaak gebruikte tool om servers te configureren. De laatste tijd zijn er echter meer concurrenten op de markt gekomen waaronder Ansible. Ansible zou ten opzichte van Puppet verschillende voordelen bieden maar aan een overschakeling van Puppet naar Ansible gaan vele werkuren vooraf. Er wordt dus beter goed nagedacht of een dergelijke omschakeling ook effectief bestaande problemen zal oplossen.

%% - Nood: waarom moest dit onderzocht worden?

Dit rapport is van toepassing voor bedrijven die worstelen met de vraag welke configuratietool het meest voor hen geschikt is en of een overschakeling economisch verantwoord is. Om deze vragen te beantwoorden wordt deze vergelijkende studie gevoerd zodat duidelijk wordt welke tool op een welbepaald vlak de betere is.

%% - Object: wat staat in dit document geschreven?
In eerste instantie zullen tekortkomingen en moeilijkheden beschreven worden die momenteel vookomen in de infrastructuur van Puppet. Gezien het feit dat dit onderziek zich afpeeld op de \gls{VRT} zal zullen deze moeilijkheden en tekortkomingen specifiek voor de \gls{VRT} zijn waar een dergelijke transitperiode van Puppet naar Ansible zich afspeelt. Dit onderzoek is dan ook gericht naar bedrijven die zichzelf hierin herkennen. Vervolgens wordt de technische werking van Ansible en Puppet geanalyseerd en gekeken naar verschillen in hun manier van werken om hetzelfde eindresultaat te bereiken. Later wordt kort aangehaald wat mogelijke gevaren en impact kunnen zijn van slecht geconfigureerde configuration management tools en er wordt afgesloten met een rapport over hoe een dergelijke transitperiode van Puppet naar Ansible precies in zijn werk gaat. In laatste instantie wordt ook het prijskaartje van beide tools aangehaald.

%% - Taak: wat heb je precies gedaan?
Dit onderzoek loopt binnen het mediabedrijf \gls{VRT} waar een dergelijke overgang plaatsvindt. De bestaande kennis en ervaringen van mensen uit de praktijk worden in dit rapport verwerkt. Door deze kennis te combineren met testen uitgevoerd op een proof of concept, die een Puppet en Ansible infrastructuur simuleren, zijn onderstaande resultaten bekomen.

%% - Resultaat: wat was het resultaat?
Zo blijkt dat Ansible door verschillende mensen wordt beschouwd als een technologie dat eenvoudig aan te leren en op te stellen is. Bovendien worden de servers die dienen geconfigureerd te worden minder belast vanwege de afwezigheid van een zogenaamde agent. Ansible en Puppet hebben een fundamenteel verschil in hun manier van communiceren. Zo verstuurt Ansible voortdurend kleine bestanden terwijl Puppet deze bundelt in twee grote reeksen. Puppet is complexer om op te stellen en het aanleren van de syntax vraagt meer moeite. Op elke server dient bovendien een Puppetagent ge\"installeerd te worden.  Puppet blijkt in de praktijk voor sommige scenario's performanter dan Ansible. Vooral bij zwaar intensieve taken tijdens deploy's scoort Puppet beter qua performantie.

%% - Conclusie: wat is/zijn de belangrijkste conclusie(s)?
Voor bedrijven die beschikken over de nodige kennis omtrend Puppet wordt een overschakeling afgeraden. Er is geen opmerkelijke winst op welk vlak dan ook. Aan bedrijven daarentegen die niet vertrouwd zijn met geautomatiseerd configuration management  wordt geadviseerd om voor Ansible te kiezen. Het is eenvoudig aan te leren en desondanks het feit dat Ansible bij een grotere hoeveelheid servers trager werkt dan Puppet, wordt hetzelfde doel bereikt. Bedrijven die zich herkennen in het profiel van de \gls{VRT} kunnen de overschakeling overwegen. Aangezien beide tools naast elkaar kunnen bestaan in dezelfde omgeving is een geleidelijke overgang perfect mogelijk. Dit stelt het bedrijf in staat om alles voldoende te testen terwijl Puppetmanifesten geleidelijk naar Ansiblerollen vertaald worden.

%% - Perspectief: blijven er nog vragen open die in de toekomst nog kunnen
%%    onderzocht worden? Wat is een mogelijk vervolg voor jouw onderzoek?
Wegens de beperkte resources en aangezien het niet van toepassing is voor de situatie van de \gls{VRT}, is er niet onderzocht hoe Ansible en Puppet zich zullen gedragen in zeer grote infrastructuren (>10.000 servers). Vermoed wordt dat Puppet beter in staat is om een grote hoeveelheid servers te configureren en dat Ansible in de moeilijkheden zou komen. Zo is Puppet in staat om de Puppetmaster op te splitsen in verschillende servers zoals PE console, Puppet DB, compile master en zo verder. Volgens hun eigen documentatie kunnen ze tot wel 20.000 servers behandelen. Ansible biedt deze mogelijkheid niet. Voorlopig is er weinig onderzoek gevoerd naar hoeveel servers dergelijke tools nu precies aankunnen. 