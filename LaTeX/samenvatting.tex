%%=============================================================================
%% Samenvatting
%%=============================================================================

%% TODO: De "abstract" of samenvatting is een kernachtige (~ 1 blz. voor een
%% thesis) synthese van het document.
%%
%% Deze aspecten moeten zeker aan bod komen:
%% - Context: waarom is dit werk belangrijk?
%% - Nood: waarom moest dit onderzocht worden?
%% - Taak: wat heb je precies gedaan?
%% - Object: wat staat in dit document geschreven?
%% - Resultaat: wat was het resultaat?
%% - Conclusie: wat is/zijn de belangrijkste conclusie(s)?
%% - Perspectief: blijven er nog vragen open die in de toekomst nog kunnen
%%    onderzocht worden? Wat is een mogelijk vervolg voor jouw onderzoek?
%%
%% LET OP! Een samenvatting is GEEN voorwoord!

%%---------- Nederlandse samenvatting -----------------------------------------
%%
%% TODO: Als je je bachelorproef in het Engels schrijft, moet je eerst een
%% Nederlandse samenvatting invoegen. Haal daarvoor onderstaande code uit
%% commentaar.
%% Wie zijn bachelorproef in het Nederlands schrijft, kan dit negeren en heel
%% deze sectie verwijderen.



%%---------- Samenvatting -----------------------------------------------------
%%
%% De samenvatting in de hoofdtaal van het document

\chapter*{Samenvatting}
% Nood: waarom moest dit onderzocht worden?
Puppet is een vaak gebruikte tool om servers te configureren. De laatste tijd zijn er echter meer concurrenten op de markt gekomen waaronder Ansible. Ansible zou ten opzichte van Puppet verschillende voordelen bieden zoals de lage leercurve. Om een overschakeling te maken van Puppet naar Ansible gaan echter vele werkuren vooraf. Er wordt dus beter goed nagedacht of een dergelijke overschakeling ook effectief bestaande problemen zal oplossen.

%% - Nood: waarom moest dit onderzocht worden?

Dit rapport is van toepassing voor bedrijven die worstelen met de vraag van welke configuratietool het meest voor hen geschikt is en of een overschakeling al het werk wel waard is. Om deze vragen te beantwoorden wordt deze vergelijkende studie gevoerd zodat duidelijk is welke tool waar in uitblinkt.

%% - Object: wat staat in dit document geschreven?
In eerste instantie zullen mogelijke problemen, tekortkomingen of moeilijkheden beschreven worden die momenteel heersen in de infrastructuur van Puppet. Dit specifiek voor de VRT, maar mogelijk dat andere bedrijven zichzelf hierin herkennen. Vervolgens wordt de technische werking van Ansible en Puppet geanalyseerd en gekeken naar hoe deze toch sterk verschillen in hun manier van werken om toch dezelfde uitkomst te hebben. Later wordt kort aagehaald wat mogelijke gevaren en impact kunnen zijn van slecht geconfigureerde configuration management tools en er wordt afgesloten met een rapport over hoe een dergelijke transitperiode van Puppet naar Ansible precies in zijn werk gaat zodat andere bedrijven weten wat hen te wachten staat. In laatste instantie wordt ook het prijskaartje van beide tools kort aangehaald.

%% - Taak: wat heb je precies gedaan?
Dit onderzoek loopt binnen het mediabedrijf VRT waar een dergelijke overgang plaatsvindt. De bestaande kennis en ervaringen van mensen uit de praktijk worden hierin verwerkt. Door deze kennis te combineren met testen uitgevoerd op een proof of concept, die namelijk een Puppet en Ansible infrastructuur simuleren, zijn de onderstaande resultaten bekomen.

%% - Resultaat: wat was het resultaat?
Zo blijkt dat Ansible door verschillende mensen wordt aanschouwd als een technologie die eenvoudig aan te leren en op te stellen is. Bovendien worden de servers die dienen geconfigureerd te worden minder belast door Ansbile vanwege de afwezigheid van een zogenaamde agent. Ansible en Puppet hebben een fundamenteel verschil in hun manier van communiceren. Zo verstuurt Ansible voortdurend kleine bestanden terwijl Puppet deze bundeld in twee grote reeksen. Puppet is verder complexer om op te stellen evenals het aanleren van de syntax. Op elke server dient bovendien een Puppetagent ge\"installeerd te worden. Ondanks dit is Puppet performanter waardoor deze sneller een consistente staat kan bereiken en hierdoor beter overweg kan met toenmende drukte.

%% - Conclusie: wat is/zijn de belangrijkste conclusie(s)?
Voor bedrijven die beschikken over de nodige kennis omtrend Puppet wordt een overschakeling afgeraden. De nodige kennis is aanwezig en er is geen opmerkelijke winst op welk vlak dan ook. Voor Bedrijven daarentegen die niet vertrouwd zijn met geautomatiseerd conifuration management  wordt geadviseerd om voor Ansible te kiezen. Het is eenvoudig aan te leren. Ondanks het feit dat Ansible bij een grotere hoeveelheid servers trager werkt dan Puppet, wordt wel hetzelfde doel bereikt.

Ook voor bedrijven die zich herkennen in het profiel van de VRT wordt een overschakeling aangeraden. Aangezien beide tools perferct naast elkaar kunnen bestaan in dezelfde omgeving is een geleidelijke overgang perfect mogelijk. Dit stelt het bedrijf in staat om alles voldoende te testen terwijl Puppetmanifesten naar Ansiblerollen vertaald worden.

%% - Perspectief: blijven er nog vragen open die in de toekomst nog kunnen
%%    onderzocht worden? Wat is een mogelijk vervolg voor jouw onderzoek?
Wegens de beperkte resources en aangezien dit niet van toepassing is voor de VRT is er niet onderzocht hoe Ansible en Puppet zich zullen gedragen in z\'e\'er grote infrastructuren. Vermoed wordt dat Puppet beter in staat is om een grote hoeveelheid servers te configureren en dat Ansible in dit soort situaties in de moeilijkheden kom. Zo is Puppet in staat om de Puppetmaster op te splitsen in verschillende servers zoals PE console, Puppet DB, compile master en zo verder. Op deze mannier is Puppet in staat om, volgens hun eigen documentatie, tot wel 20.000 servers te behandelen. Ansible biedt deze mogelijkheid niet. Voorlopig is er weinig onderzoek gevoerd naar hoeveel servers dergelijke tools nu precies aankunnen. 